\hypertarget{9f68e53c-23fe-4409-bcfe-ba4cfad5d9d9}{}
\begin{Shaded}
\begin{Highlighting}[]
\ImportTok{import}\NormalTok{ qiskit }\ImportTok{as}\NormalTok{ qk}
\ImportTok{import}\NormalTok{ numpy }\ImportTok{as}\NormalTok{ np}
\ImportTok{from}\NormalTok{ qiskit.tools.visualization }\ImportTok{import}\NormalTok{ plot\_histogram}
\ImportTok{from}\NormalTok{ qiskit }\ImportTok{import}\NormalTok{ IBMQ}
\ImportTok{from}\NormalTok{ qiskit.tools.monitor }\ImportTok{import}\NormalTok{ job\_monitor}
\ImportTok{from}\NormalTok{ qiskit.providers.aer.noise }\ImportTok{import}\NormalTok{ NoiseModel}
\ImportTok{import}\NormalTok{ matplotlib}
\ImportTok{import}\NormalTok{ matplotlib.pyplot }\ImportTok{as}\NormalTok{ plt}
\ImportTok{from}\NormalTok{ qiskit.visualization }\ImportTok{import}\NormalTok{ plot\_bloch\_multivector}
\ImportTok{from}\NormalTok{ qiskit.tools.monitor }\ImportTok{import}\NormalTok{ job\_monitor}
\OperatorTok{\%}\NormalTok{matplotlib inline}
\ImportTok{from}\NormalTok{ IPython.display }\ImportTok{import}\NormalTok{ display}
\end{Highlighting}
\end{Shaded}

\hypertarget{4f941e56-adb5-4919-bce4-6188aa7ddd7e}{}
\begin{Shaded}
\begin{Highlighting}[]
\ImportTok{import}\NormalTok{ warnings}
\NormalTok{warnings.filterwarnings(}\StringTok{\textquotesingle{}ignore\textquotesingle{}}\NormalTok{)}
\ImportTok{from}\NormalTok{ qiskit.tools.jupyter }\ImportTok{import} \OperatorTok{*}
\end{Highlighting}
\end{Shaded}

\hypertarget{f174d04f-8a68-4d83-bb0e-f34c245eacf9}{}
\begin{Shaded}
\begin{Highlighting}[]
\NormalTok{token }\OperatorTok{=} \StringTok{\textquotesingle{}b6464d13b284902ed1d1a48d2aed6bd0474c7be45011741b0fb879614419659cb722e74a046af3d5caae0398aec9bdac8843068ecbee91aff466cf3e30f3bef5\textquotesingle{}}
\ControlFlowTok{try}\NormalTok{:}
\NormalTok{    IBMQ.load\_account()}
\ControlFlowTok{except}\NormalTok{:}
\NormalTok{    qk.IBMQ.save\_account(token}\OperatorTok{=}\NormalTok{token)}
\NormalTok{    qk.IBMQ.enable\_account(token)}
\NormalTok{provider }\OperatorTok{=}\NormalTok{ IBMQ.get\_provider(hub}\OperatorTok{=}\StringTok{"ibm{-}q"}\NormalTok{, group}\OperatorTok{=}\StringTok{"open"}\NormalTok{, project}\OperatorTok{=}\StringTok{"main"}\NormalTok{)}
\NormalTok{backend }\OperatorTok{=}\NormalTok{ provider.get\_backend(}\StringTok{"ibmq\_armonk"}\NormalTok{)}
\end{Highlighting}
\end{Shaded}

\hypertarget{4b373e23-a658-4382-b744-95ed4f4b8214}{}
\begin{Shaded}
\begin{Highlighting}[]
\NormalTok{backend\_config }\OperatorTok{=}\NormalTok{ backend.configuration()}
\ControlFlowTok{assert}\NormalTok{ backend\_config.open\_pulse, }\StringTok{"Backend doesn\textquotesingle{}t support Pulse"}
\end{Highlighting}
\end{Shaded}

\hypertarget{69fe775c-142b-4dcb-b68b-47940c7e72a3}{}
\begin{Shaded}
\begin{Highlighting}[]
\NormalTok{dt }\OperatorTok{=}\NormalTok{ backend\_config.dt}
\BuiltInTok{print}\NormalTok{(}\SpecialStringTok{f"Sampling time: }\SpecialCharTok{\{}\NormalTok{dt}\OperatorTok{*}\FloatTok{1e9}\SpecialCharTok{\}}\SpecialStringTok{ ns"}\NormalTok{) }
\NormalTok{backend\_defaults }\OperatorTok{=}\NormalTok{ backend.defaults()}
\BuiltInTok{print}\NormalTok{(backend\_defaults)}
\end{Highlighting}
\end{Shaded}

\begin{verbatim}
Sampling time: 0.2222222222222222 ns
<PulseDefaults(<InstructionScheduleMap(1Q instructions:
  q0: {'id', 'u1', 'measure', 'rz', 'sx', 'x', 'u3', 'u2'}
Multi qubit instructions:
)>Qubit Frequencies [GHz]
[4.9716805437955225]
Measurement Frequencies [GHz]
[6.993370669] )>
\end{verbatim}

\hypertarget{fd6aa8b0-c714-4255-9b74-a3b750c321e6}{}
\hypertarget{setteing-up-frequency-sweep}{%
\section{Setteing up Frequency
Sweep}\label{setteing-up-frequency-sweep}}

\leavevmode\vadjust pre{\hypertarget{af158817-2c73-4e66-b679-95f461d51480}{}}%
We will first condunct Frequency sweep over a large range to identify
our transiton energies

\hypertarget{d5d7ed8b-5e53-4b6c-bada-e71c0d60b18e}{}
\hypertarget{initial-frequency-0-1}{%
\section{Initial Frequency
0-\textgreater1}\label{initial-frequency-0-1}}

\hypertarget{a152d6f7-5311-4312-8ce7-829a719c7cba}{}
\begin{Shaded}
\begin{Highlighting}[]
\ImportTok{import}\NormalTok{ numpy }\ImportTok{as}\NormalTok{ np}

\CommentTok{\# unit conversion factors {-}\textgreater{} all backend properties returned in SI (Hz, sec, etc.)}
\NormalTok{GHz }\OperatorTok{=} \FloatTok{1.0e9} \CommentTok{\# Gigahertz}
\NormalTok{MHz }\OperatorTok{=} \FloatTok{1.0e6} \CommentTok{\# Megahertz}
\NormalTok{us }\OperatorTok{=} \FloatTok{1.0e{-}6} \CommentTok{\# Microseconds}
\NormalTok{ns }\OperatorTok{=} \FloatTok{1.0e{-}9} \CommentTok{\# Nanoseconds}

\CommentTok{\# We will find the qubit frequency for the following qubit.}
\NormalTok{qubit }\OperatorTok{=} \DecValTok{0}
\CommentTok{\# We will define memory slot channel 0.}
\NormalTok{mem\_slot }\OperatorTok{=} \DecValTok{0}

\CommentTok{\# The sweep will be centered around the estimated qubit frequency.}
\NormalTok{center\_frequency\_Hz }\OperatorTok{=}\NormalTok{ backend\_defaults.qubit\_freq\_est[qubit]        }\CommentTok{\# The default frequency is given in Hz}
                                                                    \CommentTok{\# warning: this will change in a future release}
\BuiltInTok{print}\NormalTok{(}\SpecialStringTok{f"Qubit }\SpecialCharTok{\{}\NormalTok{qubit}\SpecialCharTok{\}}\SpecialStringTok{ has an estimated frequency of }\SpecialCharTok{\{}\NormalTok{center\_frequency\_Hz }\OperatorTok{/}\NormalTok{ GHz}\SpecialCharTok{\}}\SpecialStringTok{ GHz."}\NormalTok{)}

\CommentTok{\# scale factor to remove factors of 10 from the data}
\NormalTok{scale\_factor }\OperatorTok{=} \FloatTok{1e{-}15}

\CommentTok{\# We will sweep 40 MHz around the estimated frequency}
\NormalTok{frequency\_span\_Hz }\OperatorTok{=} \DecValTok{20} \OperatorTok{*}\NormalTok{ MHz}
\CommentTok{\# in steps of 1 MHz.}
\NormalTok{frequency\_step\_Hz }\OperatorTok{=} \DecValTok{1} \OperatorTok{*}\NormalTok{ MHz}
\NormalTok{a}\OperatorTok{=}\DecValTok{0}
\CommentTok{\# We will sweep 20 MHz above and 20 MHz below the estimated frequency}
\NormalTok{frequency\_min }\OperatorTok{=}\NormalTok{ center\_frequency\_Hz }\OperatorTok{{-}}\NormalTok{ frequency\_span\_Hz }\OperatorTok{/} \DecValTok{2}
\NormalTok{frequency\_max }\OperatorTok{=}\NormalTok{ center\_frequency\_Hz }\OperatorTok{+}\NormalTok{ frequency\_span\_Hz }\OperatorTok{/} \DecValTok{2}
\CommentTok{\# Construct an np array of the frequencies for our experiment}
\NormalTok{frequencies\_GHz }\OperatorTok{=}\NormalTok{ np.arange((frequency\_min }\OperatorTok{/}\NormalTok{ GHz)}\OperatorTok{{-}}\NormalTok{a, }
\NormalTok{                            (frequency\_max }\OperatorTok{/}\NormalTok{ GHz)}\OperatorTok{+}\NormalTok{a, }
\NormalTok{                            frequency\_step\_Hz }\OperatorTok{/}\NormalTok{ GHz)}

\BuiltInTok{print}\NormalTok{(}\SpecialStringTok{f"The sweep will go from }\SpecialCharTok{\{}\NormalTok{frequency\_min }\OperatorTok{/}\NormalTok{ GHz}\SpecialCharTok{\}}\SpecialStringTok{ GHz to }\SpecialCharTok{\{}\NormalTok{frequency\_max }\OperatorTok{/}\NormalTok{ GHz}\SpecialCharTok{\}}\SpecialStringTok{ GHz }\CharTok{\textbackslash{}}
\SpecialStringTok{in steps of }\SpecialCharTok{\{}\NormalTok{frequency\_step\_Hz }\OperatorTok{/}\NormalTok{ MHz}\SpecialCharTok{\}}\SpecialStringTok{ MHz."}\NormalTok{)}
\BuiltInTok{print}\NormalTok{(}\BuiltInTok{len}\NormalTok{(frequencies\_GHz))}
\end{Highlighting}
\end{Shaded}

\begin{verbatim}
Qubit 0 has an estimated frequency of 4.9716805437955225 GHz.
The sweep will go from 4.961680543795523 GHz to 4.981680543795522 GHz in steps of 1.0 MHz.
20
\end{verbatim}

\leavevmode\vadjust pre{\hypertarget{e5f59b59-c64c-4e3d-ac01-b36e452d40cd}{}}%
We will create a pulse schedule by defining this frequency as a
parameter using the parameter class. First, we will set the required
values duration, sigma, and channel.

Then we will set the pulse flow so that the specified pulses are
executed sequentially. We will define the pulse frequency, the pulse
used in the experiment, and the measurement pulse. Here, the pulse used
in the experiment specifies the drive pulse, which is a Gaussian pulse.

At each frequency, we will send a drive pulse of that frequency to the
qubit and measure immediately after the pulse.

\hypertarget{38518258-a0b8-4a6f-817b-38fe07f9f5db}{}
\begin{Shaded}
\begin{Highlighting}[]
\CommentTok{\# samples need to be multiples of 16}
\KeywordTok{def}\NormalTok{ get\_closest\_multiple\_of\_16(num):}
    \ControlFlowTok{return} \BuiltInTok{int}\NormalTok{(num }\OperatorTok{+} \DecValTok{8}\NormalTok{ ) }\OperatorTok{{-}}\NormalTok{ (}\BuiltInTok{int}\NormalTok{(num }\OperatorTok{+} \DecValTok{8}\NormalTok{ ) }\OperatorTok{\%} \DecValTok{16}\NormalTok{)}
\end{Highlighting}
\end{Shaded}

\hypertarget{52261d49-5a56-41bb-bc4b-0b0c325b4cc2}{}
\begin{Shaded}
\begin{Highlighting}[]
\CommentTok{\# Convert seconds to dt}
\KeywordTok{def}\NormalTok{ get\_dt\_from(sec):}
    \ControlFlowTok{return}\NormalTok{ get\_closest\_multiple\_of\_16(sec}\OperatorTok{/}\NormalTok{dt)}
\end{Highlighting}
\end{Shaded}

\hypertarget{47dc8417-515a-4074-b7fe-a6d577e8f676}{}
\begin{Shaded}
\begin{Highlighting}[]
\ImportTok{from}\NormalTok{ qiskit }\ImportTok{import}\NormalTok{ pulse                  }\CommentTok{\# This is where we access all of our Pulse features!}
\ImportTok{from}\NormalTok{ qiskit.circuit }\ImportTok{import}\NormalTok{ Parameter      }\CommentTok{\# This is Parameter Class for variable parameters.}


\CommentTok{\# Drive pulse parameters (us = microseconds)}
\NormalTok{drive\_sigma\_sec }\OperatorTok{=} \FloatTok{0.015} \OperatorTok{*}\NormalTok{ us }\OperatorTok{*}\DecValTok{4}                         \CommentTok{\# This determines the actual width of the gaussian}
\NormalTok{drive\_duration\_sec }\OperatorTok{=}\NormalTok{ drive\_sigma\_sec }\OperatorTok{*} \DecValTok{8}                \CommentTok{\# This is a truncating parameter, because gaussians don\textquotesingle{}t have }
                                                        \CommentTok{\# a natural finite length}
\NormalTok{drive\_amp }\OperatorTok{=} \FloatTok{0.5}

\NormalTok{frequencies\_Hz }\OperatorTok{=}\NormalTok{ frequencies\_GHz}\OperatorTok{*}\NormalTok{GHz}
\CommentTok{\# Create the base schedule}
\CommentTok{\# Start with drive pulse acting on the drive channel}
\NormalTok{freq }\OperatorTok{=}\NormalTok{ Parameter(}\StringTok{\textquotesingle{}freq\textquotesingle{}}\NormalTok{)}
\CommentTok{\#time = Parameter(\textquotesingle{}time\textquotesingle{})}
\ControlFlowTok{with}\NormalTok{ pulse.build(backend}\OperatorTok{=}\NormalTok{backend, default\_alignment}\OperatorTok{=}\StringTok{\textquotesingle{}sequential\textquotesingle{}}\NormalTok{, name}\OperatorTok{=}\StringTok{\textquotesingle{}Frequency sweep\textquotesingle{}}\NormalTok{) }\ImportTok{as}\NormalTok{ sweep\_sched:}
\NormalTok{    drive\_duration }\OperatorTok{=}\NormalTok{ get\_closest\_multiple\_of\_16(pulse.seconds\_to\_samples(drive\_duration\_sec))}
\NormalTok{    drive\_sigma }\OperatorTok{=}\NormalTok{ pulse.seconds\_to\_samples(drive\_sigma\_sec)}
\NormalTok{    drive\_chan }\OperatorTok{=}\NormalTok{ pulse.drive\_channel(qubit)}
\NormalTok{    pulse.set\_frequency(freq, drive\_chan)}
    \CommentTok{\# Drive pulse samples}
    \BuiltInTok{print}\NormalTok{(drive\_sigma)}
\NormalTok{    pulse.play(pulse.Gaussian(duration}\OperatorTok{=}\NormalTok{drive\_duration,}
\NormalTok{                              sigma}\OperatorTok{=}\NormalTok{drive\_sigma,}
\NormalTok{                              amp}\OperatorTok{=}\NormalTok{drive\_amp,}
\NormalTok{                              name}\OperatorTok{=}\StringTok{\textquotesingle{}freq\_sweep\_excitation\_pulse\textquotesingle{}}\NormalTok{), drive\_chan)}
    \CommentTok{\# Define our measurement pulse}
\NormalTok{    pulse.measure(qubits}\OperatorTok{=}\NormalTok{[qubit], registers}\OperatorTok{=}\NormalTok{[pulse.MemorySlot(mem\_slot)])}
        

\CommentTok{\# Create the frequency settings for the sweep (MUST BE IN HZ)}

\NormalTok{times}\OperatorTok{=}\NormalTok{ [}\DecValTok{1}\NormalTok{,}\DecValTok{2}\NormalTok{,}\DecValTok{3}\NormalTok{,}\DecValTok{4}\NormalTok{,}\DecValTok{5}\NormalTok{]}
\NormalTok{schedules }\OperatorTok{=}\NormalTok{ [sweep\_sched.assign\_parameters(\{freq: f\}, inplace}\OperatorTok{=}\VariableTok{False}\NormalTok{) }\ControlFlowTok{for}\NormalTok{ f }\KeywordTok{in}\NormalTok{ frequencies\_Hz]}
\BuiltInTok{print}\NormalTok{(times)}
\end{Highlighting}
\end{Shaded}

\begin{verbatim}
270
[1, 2, 3, 4, 5]
\end{verbatim}

\leavevmode\vadjust pre{\hypertarget{2f62ab23-9324-4bc7-b925-d7ae5afac43c}{}}%
As a sanity check, it's always a good idea to look at the pulse
schedule. This is done using \texttt{schedule.draw()} as shown below.

\hypertarget{40689979-3fa6-46b6-a7fd-b16b77dec2e7}{}
\begin{Shaded}
\begin{Highlighting}[]
\NormalTok{schedules[}\DecValTok{0}\NormalTok{].draw(backend}\OperatorTok{=}\NormalTok{backend)}
\NormalTok{schedules[}\DecValTok{5}\NormalTok{].draw(backend}\OperatorTok{=}\NormalTok{backend)}
\end{Highlighting}
\end{Shaded}

\includegraphics{4970572ab007753a76d5051c43bcb926db605088.png}

\leavevmode\vadjust pre{\hypertarget{b69cbea2-f4e3-47ae-bc87-ec0b9fa5fba4}{}}%
We request that each schedule (each point in our frequency sweep) is
repeated \texttt{num\_shots\_per\_frequency} times in order to get a
good estimate of the qubit response.

We also specify measurement settings. \texttt{meas\_level=0} returns raw
data (an array of complex values per shot), \texttt{meas\_level=1}
returns kerneled data (one complex value per shot), and
\texttt{meas\_level=2} returns classified data (a 0 or 1 bit per shot).
We choose \texttt{meas\_level=1} to replicate what we would be working
with if we were in the lab, and hadn't yet calibrated the discriminator
to classify 0s and 1s. We ask for the
\texttt{\textquotesingle{}avg\textquotesingle{}} of the results, rather
than each shot individually.

\leavevmode\vadjust pre{\hypertarget{37da8053-0d55-4383-8c6b-db2110828733}{}}%
You may see yet another unit change warning, we can safely ignore this.
Finally, we can run the assembled program on the backend using:

\hypertarget{ed65f46e-f38e-47ee-b973-d2f110666583}{}
\begin{Shaded}
\begin{Highlighting}[]
\NormalTok{num\_shots\_per\_frequency }\OperatorTok{=} \DecValTok{1024}

\NormalTok{job }\OperatorTok{=}\NormalTok{ backend.run(schedules, }
\NormalTok{                  meas\_level}\OperatorTok{=}\DecValTok{1}\NormalTok{, }
\NormalTok{                  meas\_return}\OperatorTok{=}\StringTok{\textquotesingle{}avg\textquotesingle{}}\NormalTok{, }
\NormalTok{                  shots}\OperatorTok{=}\NormalTok{num\_shots\_per\_frequency)}
\end{Highlighting}
\end{Shaded}

\leavevmode\vadjust pre{\hypertarget{2702f65a-834d-40b4-b9cf-0c3aa641cfce}{}}%
It is always a good idea to monitor the job status by using
\texttt{job\_monitor()}

\hypertarget{4f1aa3f2-59f1-403d-9742-42897deb0deb}{}
\begin{Shaded}
\begin{Highlighting}[]
\ImportTok{from}\NormalTok{ qiskit.tools.monitor }\ImportTok{import}\NormalTok{ job\_monitor}
\NormalTok{job\_monitor(job)}
\end{Highlighting}
\end{Shaded}

\begin{verbatim}
Job Status: job has successfully run
\end{verbatim}

\leavevmode\vadjust pre{\hypertarget{2b3f4ce9-e991-4b35-81dc-03a620a07776}{}}%
Once the job is run, the results can be retrieved using:

\hypertarget{6350f277-6624-4758-8df3-13b6cea23cad}{}
\begin{Shaded}
\begin{Highlighting}[]
\NormalTok{frequency\_sweep\_results}\OperatorTok{=}\NormalTok{ job.result(timeout}\OperatorTok{=}\DecValTok{120}\NormalTok{) }\CommentTok{\# timeout parameter set to 120 second}
\end{Highlighting}
\end{Shaded}

\leavevmode\vadjust pre{\hypertarget{89d06bbb-6a4e-4da6-ad8f-b6bb73724fc0}{}}%
We will extract the results and plot them using \texttt{matplotlib}:

\hypertarget{48cdd8c7-6fc2-4774-99ee-71b370de0ab0}{}
\begin{Shaded}
\begin{Highlighting}[]
\ImportTok{import}\NormalTok{ matplotlib.pyplot }\ImportTok{as}\NormalTok{ plt}

\NormalTok{sweep\_values }\OperatorTok{=}\NormalTok{ []}
\ControlFlowTok{for}\NormalTok{ i }\KeywordTok{in} \BuiltInTok{range}\NormalTok{(}\BuiltInTok{len}\NormalTok{(frequency\_sweep\_results.results)):}
    \CommentTok{\# Get the results from the ith experiment}
\NormalTok{    res }\OperatorTok{=}\NormalTok{ frequency\_sweep\_results.get\_memory(i)}\OperatorTok{*}\NormalTok{scale\_factor}
    \CommentTok{\# Get the results for \textasciigrave{}qubit\textasciigrave{} from this experiment}
\NormalTok{    sweep\_values.append(res[qubit])}
    \ControlFlowTok{if}\NormalTok{ (i}\OperatorTok{\%}\DecValTok{4}\OperatorTok{==}\DecValTok{0}\NormalTok{): }\BuiltInTok{print}\NormalTok{((frequency\_sweep\_results.get\_memory(i)}\OperatorTok{*}\NormalTok{scale\_factor)[}\DecValTok{0}\NormalTok{])}

\NormalTok{plt.scatter(frequencies\_GHz, np.real(sweep\_values), color}\OperatorTok{=}\StringTok{\textquotesingle{}black\textquotesingle{}}\NormalTok{) }\CommentTok{\# plot real part of sweep values}
\NormalTok{plt.xlim([}\BuiltInTok{min}\NormalTok{(frequencies\_GHz), }\BuiltInTok{max}\NormalTok{(frequencies\_GHz)])}
\NormalTok{plt.xlabel(}\StringTok{"Frequency [GHz]"}\NormalTok{)}
\NormalTok{plt.ylabel(}\StringTok{"Measured signal [a.u.]"}\NormalTok{)}
\NormalTok{plt.show()}
\end{Highlighting}
\end{Shaded}

\begin{verbatim}
(-1.25892672094208-1.211957361246208j)
(-1.233851225473024-1.228896443826176j)
(-0.961655525605376-1.467631395667968j)
(-0.960599836065792-1.453804553764864j)
(-1.232183033331712-1.222821078368256j)
\end{verbatim}

\includegraphics{6187b313f603fcfb265ec92615122cb8a37ecef3.png}

\leavevmode\vadjust pre{\hypertarget{b0bcfde0-e83b-4d41-a8d7-9d56dce568e0}{}}%
As you can see above, the peak near the center corresponds to the
location of the qubit frequency. The signal shows power-broadening,
which is a signature that we are able to drive the qubit off-resonance
as we get close to the center frequency. To get the value of the peak
frequency, we will fit the values to a resonance response curve, which
is typically a Lorentzian shape.

\hypertarget{2f0ad1be-c269-4b57-b278-554fc5f351f4}{}
\begin{Shaded}
\begin{Highlighting}[]
\ImportTok{from}\NormalTok{ scipy.optimize }\ImportTok{import}\NormalTok{ curve\_fit}

\KeywordTok{def}\NormalTok{ fit\_function(x\_values, y\_values, function, init\_params):}
\NormalTok{    fitparams, conv }\OperatorTok{=}\NormalTok{ curve\_fit(function, x\_values, y\_values, init\_params)}
\NormalTok{    y\_fit }\OperatorTok{=}\NormalTok{ function(x\_values, }\OperatorTok{*}\NormalTok{fitparams)}
    \ControlFlowTok{return}\NormalTok{ fitparams, y\_fit}
\end{Highlighting}
\end{Shaded}

\hypertarget{7e7e85aa-bff3-473c-aa4e-70dae30f1af1}{}
\begin{Shaded}
\begin{Highlighting}[]
\NormalTok{fit\_params, y\_fit }\OperatorTok{=}\NormalTok{ fit\_function(frequencies\_GHz,}
\NormalTok{                                 np.real(sweep\_values), }
                                 \KeywordTok{lambda}\NormalTok{ x, A, q\_freq, B, C: ((A}\OperatorTok{/}\NormalTok{np.pi)}\OperatorTok{*}\NormalTok{(B}\OperatorTok{/}\NormalTok{ ((x}\OperatorTok{{-}}\NormalTok{q\_freq)}\OperatorTok{**}\DecValTok{2}\OperatorTok{+}\NormalTok{B}\OperatorTok{**}\DecValTok{2}\NormalTok{))}\OperatorTok{+}\NormalTok{C),}
\NormalTok{                                 [}\FloatTok{0.003}\NormalTok{, }\FloatTok{4.9723}\NormalTok{, }\FloatTok{0.003}\NormalTok{,}\OperatorTok{{-}}\FloatTok{1.25}\NormalTok{] }\CommentTok{\# initial parameters for curve\_fit}
\NormalTok{                                )}
\NormalTok{plt.scatter(frequencies\_GHz, np.real(sweep\_values), color}\OperatorTok{=}\StringTok{\textquotesingle{}black\textquotesingle{}}\NormalTok{)}
\NormalTok{plt.plot(frequencies\_GHz, y\_fit, color}\OperatorTok{=}\StringTok{\textquotesingle{}red\textquotesingle{}}\NormalTok{)}
\NormalTok{plt.xlim([}\BuiltInTok{min}\NormalTok{(frequencies\_GHz), }\BuiltInTok{max}\NormalTok{(frequencies\_GHz)])}

\NormalTok{plt.xlabel(}\StringTok{"Frequency [GHz]"}\NormalTok{)}
\NormalTok{plt.ylabel(}\StringTok{"Measured Signal [a.u.]"}\NormalTok{)}
\NormalTok{plt.show()}
\end{Highlighting}
\end{Shaded}

\includegraphics{24dcb0cc9171a5f1bfd83676ae339798128228b5.png}

\hypertarget{27e60bc3-cbc9-47cb-8b16-249429e6683e}{}
\begin{Shaded}
\begin{Highlighting}[]
\NormalTok{A, rough\_qubit\_frequency, B, C }\OperatorTok{=}\NormalTok{ fit\_params}
\NormalTok{rough\_qubit\_frequency }\OperatorTok{=}\NormalTok{ rough\_qubit\_frequency}\OperatorTok{*}\NormalTok{GHz }\CommentTok{\# make sure qubit freq is in Hz}
\BuiltInTok{print}\NormalTok{(}\SpecialStringTok{f"We\textquotesingle{}ve updated our qubit frequency estimate from "}
      \SpecialStringTok{f"}\SpecialCharTok{\{}\BuiltInTok{round}\NormalTok{(backend\_defaults.qubit\_freq\_est[qubit] }\OperatorTok{/}\NormalTok{ GHz, }\DecValTok{5}\NormalTok{)}\SpecialCharTok{\}}\SpecialStringTok{ GHz to }\SpecialCharTok{\{}\BuiltInTok{round}\NormalTok{(rough\_qubit\_frequency}\OperatorTok{/}\NormalTok{GHz, }\DecValTok{5}\NormalTok{)}\SpecialCharTok{\}}\SpecialStringTok{ GHz."}\NormalTok{)}
\CommentTok{\#print(fit\_params)}
\end{Highlighting}
\end{Shaded}

\begin{verbatim}
We've updated our qubit frequency estimate from 4.97168 GHz to 4.97174 GHz.
\end{verbatim}

\hypertarget{e324ac92-bb33-4a86-b14a-919fae7eec4b}{}
\hypertarget{using-rabi-to-calibrate-the-pi-pulse-for-0---1}{%
\section{\texorpdfstring{Using Rabi to Calibrate the \(\pi\) Pulse for
\textbar0\textgreater{} -\textgreater{}
\textbar1\textgreater{}}{Using Rabi to Calibrate the \textbackslash pi Pulse for \textbar0\textgreater{} -\textgreater{} \textbar1\textgreater{}}}\label{using-rabi-to-calibrate-the-pi-pulse-for-0---1}}

\hypertarget{8346b765-b22b-47eb-b9ea-30ce7e54986a}{}
\begin{Shaded}
\begin{Highlighting}[]
\CommentTok{\# This experiment uses these values from the previous experiment:}
    \CommentTok{\# \textasciigrave{}qubit\textasciigrave{},}
    \CommentTok{\# \textasciigrave{}mem\_slot\textasciigrave{}, and}
    \CommentTok{\# \textasciigrave{}rough\_qubit\_frequency\textasciigrave{}.}

\CommentTok{\# Rabi experiment parameters}
\NormalTok{num\_rabi\_points }\OperatorTok{=} \DecValTok{50}

\CommentTok{\# Drive amplitude values to iterate over: 50 amplitudes evenly spaced from 0 to 0.75}
\NormalTok{drive\_amp\_min }\OperatorTok{=} \OperatorTok{{-}}\DecValTok{1}
\NormalTok{drive\_amp\_max }\OperatorTok{=} \DecValTok{1}
\NormalTok{drive\_amps }\OperatorTok{=}\NormalTok{ np.linspace(drive\_amp\_min, drive\_amp\_max, num\_rabi\_points)}
\end{Highlighting}
\end{Shaded}

\hypertarget{ef28af9e-5a1a-4e78-b4fa-e882ed9afaf1}{}
\begin{Shaded}
\begin{Highlighting}[]
\CommentTok{\# Build the Rabi experiments:}
\CommentTok{\#    A drive pulse at the qubit frequency, followed by a measurement,}
\CommentTok{\#    where we vary the drive amplitude each time.}

\NormalTok{drive\_amp }\OperatorTok{=}\NormalTok{ Parameter(}\StringTok{\textquotesingle{}drive\_amp\textquotesingle{}}\NormalTok{)}
\ControlFlowTok{with}\NormalTok{ pulse.build(backend}\OperatorTok{=}\NormalTok{backend, default\_alignment}\OperatorTok{=}\StringTok{\textquotesingle{}sequential\textquotesingle{}}\NormalTok{, name}\OperatorTok{=}\StringTok{\textquotesingle{}Rabi Experiment\textquotesingle{}}\NormalTok{) }\ImportTok{as}\NormalTok{ rabi\_sched:}
\NormalTok{    drive\_duration }\OperatorTok{=}\NormalTok{ get\_closest\_multiple\_of\_16(pulse.seconds\_to\_samples(drive\_duration\_sec))}
\NormalTok{    drive\_sigma }\OperatorTok{=}\NormalTok{ pulse.seconds\_to\_samples(drive\_sigma\_sec)}
\NormalTok{    drive\_chan }\OperatorTok{=}\NormalTok{ pulse.drive\_channel(qubit)}
\NormalTok{    pulse.set\_frequency(rough\_qubit\_frequency, drive\_chan)}
\NormalTok{    pulse.play(pulse.Gaussian(duration}\OperatorTok{=}\NormalTok{drive\_duration,}
\NormalTok{                              amp}\OperatorTok{=}\NormalTok{drive\_amp,}
\NormalTok{                              sigma}\OperatorTok{=}\NormalTok{drive\_sigma,}
\NormalTok{                              name}\OperatorTok{=}\StringTok{\textquotesingle{}Rabi Pulse\textquotesingle{}}\NormalTok{), drive\_chan)}
\NormalTok{    pulse.measure(qubits}\OperatorTok{=}\NormalTok{[qubit], registers}\OperatorTok{=}\NormalTok{[pulse.MemorySlot(mem\_slot)])}

\NormalTok{rabi\_schedules }\OperatorTok{=}\NormalTok{ [rabi\_sched.assign\_parameters(\{drive\_amp: a\}, inplace}\OperatorTok{=}\VariableTok{False}\NormalTok{) }\ControlFlowTok{for}\NormalTok{ a }\KeywordTok{in}\NormalTok{ drive\_amps]}
\end{Highlighting}
\end{Shaded}

\leavevmode\vadjust pre{\hypertarget{2c04bf7a-afe0-4c09-81ed-381b64e8aaaf}{}}%
The schedule will look essentially the same as the frequency sweep
experiment. The only difference is that we are running a set of
experiments which vary the amplitude of the drive pulse, rather than its
modulation frequency.

\hypertarget{03cdd9a4-f3c3-44af-a07a-6b7133e20b09}{}
\begin{Shaded}
\begin{Highlighting}[]
\NormalTok{rabi\_schedules[}\DecValTok{3}\NormalTok{].draw(backend}\OperatorTok{=}\NormalTok{backend)}
\end{Highlighting}
\end{Shaded}

\includegraphics{22a6e553ff023f018d3794258648d0348dbcba91.png}

\hypertarget{be62f139-c5cb-4592-9767-4768e145dd3f}{}
\begin{Shaded}
\begin{Highlighting}[]
\NormalTok{num\_shots\_per\_point }\OperatorTok{=} \DecValTok{1024}

\NormalTok{job }\OperatorTok{=}\NormalTok{ backend.run(rabi\_schedules, }
\NormalTok{                  meas\_level}\OperatorTok{=}\DecValTok{1}\NormalTok{, }
\NormalTok{                  meas\_return}\OperatorTok{=}\StringTok{\textquotesingle{}avg\textquotesingle{}}\NormalTok{, }
\NormalTok{                  shots}\OperatorTok{=}\NormalTok{num\_shots\_per\_point)}

\NormalTok{job\_monitor(job)}
\end{Highlighting}
\end{Shaded}

\begin{verbatim}
Job Status: job has successfully run
\end{verbatim}

\hypertarget{88c2d2be-f3c7-4ccc-b06e-0394779a9d66}{}
\begin{Shaded}
\begin{Highlighting}[]
\NormalTok{rabi\_results }\OperatorTok{=}\NormalTok{ job.result(timeout}\OperatorTok{=}\DecValTok{120}\NormalTok{)}
\end{Highlighting}
\end{Shaded}

\leavevmode\vadjust pre{\hypertarget{7d732f65-2284-4faa-9124-0853c99bf493}{}}%
Now that we have our results, we will extract them and fit them to a
sinusoidal curve. For the range of drive amplitudes we selected, we
expect that we will rotate the qubit several times completely around the
Bloch sphere, starting from \(|0\rangle\). The amplitude of this
sinusoid tells us the fraction of the shots at that Rabi drive amplitude
which yielded the \(|1\rangle\) state. We want to find the drive
amplitude needed for the signal to oscillate from a maximum (all
\(|0\rangle\) state) to a minimum (all \(|1\rangle\) state) -\/- this
gives the calibrated amplitude that enacts a \(\pi\) pulse.

\hypertarget{dc6dccd4-19af-4414-93bd-631427907bc6}{}
\begin{Shaded}
\begin{Highlighting}[]
\CommentTok{\# center data around 0}
\KeywordTok{def}\NormalTok{ baseline\_remove(values):}
    \ControlFlowTok{return}\NormalTok{ np.array(values) }\OperatorTok{{-}}\NormalTok{ np.mean(values)}
\end{Highlighting}
\end{Shaded}

\hypertarget{339c9822-8bfd-49bf-8c2d-edef22b5b90c}{}
\begin{Shaded}
\begin{Highlighting}[]
\NormalTok{rabi\_values }\OperatorTok{=}\NormalTok{ []}
\ControlFlowTok{for}\NormalTok{ i }\KeywordTok{in} \BuiltInTok{range}\NormalTok{(num\_rabi\_points):}
    \CommentTok{\# Get the results for \textasciigrave{}qubit\textasciigrave{} from the ith experiment}
\NormalTok{    rabi\_values.append(rabi\_results.get\_memory(i)[qubit] }\OperatorTok{*}\NormalTok{ scale\_factor)}

\NormalTok{rabi\_values }\OperatorTok{=}\NormalTok{ np.real(baseline\_remove(rabi\_values))}

\NormalTok{plt.xlabel(}\StringTok{"Drive amp [a.u.]"}\NormalTok{)}
\NormalTok{plt.ylabel(}\StringTok{"Measured signal [a.u.]"}\NormalTok{)}
\NormalTok{plt.scatter(drive\_amps, rabi\_values, color}\OperatorTok{=}\StringTok{\textquotesingle{}black\textquotesingle{}}\NormalTok{) }\CommentTok{\# plot real part of Rabi values}
\NormalTok{plt.show()}
\end{Highlighting}
\end{Shaded}

\includegraphics{35ec097a0ffaf476676eb0b8f065969d0426efae.png}

\hypertarget{f98a67bc-bff6-44d9-a3ff-c1bab43c429f}{}
\begin{Shaded}
\begin{Highlighting}[]
\NormalTok{fit\_params, y\_fit }\OperatorTok{=}\NormalTok{ fit\_function(drive\_amps,}
\NormalTok{                                 rabi\_values, }
                                 \KeywordTok{lambda}\NormalTok{ x, A, B, drive\_period, phi: (A}\OperatorTok{*}\NormalTok{np.cos(}\DecValTok{2}\OperatorTok{*}\NormalTok{np.pi}\OperatorTok{*}\NormalTok{x}\OperatorTok{/}\NormalTok{drive\_period }\OperatorTok{{-}}\NormalTok{ phi) }\OperatorTok{+}\NormalTok{ B),}
\NormalTok{                                 [}\FloatTok{0.2}\NormalTok{, }\DecValTok{0}\NormalTok{, }\FloatTok{0.4}\NormalTok{, np.pi}\OperatorTok{/}\DecValTok{2}\NormalTok{])}

\NormalTok{plt.scatter(drive\_amps, rabi\_values, color}\OperatorTok{=}\StringTok{\textquotesingle{}black\textquotesingle{}}\NormalTok{)}
\NormalTok{plt.plot(drive\_amps, y\_fit, color}\OperatorTok{=}\StringTok{\textquotesingle{}red\textquotesingle{}}\NormalTok{)}
\BuiltInTok{print}\NormalTok{(fit\_params)}
\NormalTok{drive\_period }\OperatorTok{=}\NormalTok{ fit\_params[}\DecValTok{2}\NormalTok{] }\CommentTok{\# get period of rabi oscillation}

\NormalTok{plt.axvline(}\DecValTok{0}\NormalTok{, color}\OperatorTok{=}\StringTok{\textquotesingle{}red\textquotesingle{}}\NormalTok{, linestyle}\OperatorTok{=}\StringTok{\textquotesingle{}{-}{-}\textquotesingle{}}\NormalTok{)}
\NormalTok{plt.axvline(drive\_period}\OperatorTok{/}\DecValTok{2}\NormalTok{, color}\OperatorTok{=}\StringTok{\textquotesingle{}red\textquotesingle{}}\NormalTok{, linestyle}\OperatorTok{=}\StringTok{\textquotesingle{}{-}{-}\textquotesingle{}}\NormalTok{)}
\NormalTok{plt.annotate(}\StringTok{""}\NormalTok{, xy}\OperatorTok{=}\NormalTok{(}\DecValTok{0}\NormalTok{, }\DecValTok{0}\NormalTok{), xytext}\OperatorTok{=}\NormalTok{(drive\_period}\OperatorTok{/}\DecValTok{2}\NormalTok{,}\DecValTok{0}\NormalTok{), arrowprops}\OperatorTok{=}\BuiltInTok{dict}\NormalTok{(arrowstyle}\OperatorTok{=}\StringTok{"\textless{}{-}\textgreater{}"}\NormalTok{, color}\OperatorTok{=}\StringTok{\textquotesingle{}red\textquotesingle{}}\NormalTok{))}
\NormalTok{plt.annotate(}\StringTok{"$\textbackslash{}pi$"}\NormalTok{, xy}\OperatorTok{=}\NormalTok{(drive\_period}\OperatorTok{/}\DecValTok{2}\OperatorTok{{-}}\FloatTok{0.03}\NormalTok{, }\FloatTok{0.1}\NormalTok{), color}\OperatorTok{=}\StringTok{\textquotesingle{}red\textquotesingle{}}\NormalTok{)}

\NormalTok{plt.xlabel(}\StringTok{"Drive amp [a.u.]"}\NormalTok{, fontsize}\OperatorTok{=}\DecValTok{15}\NormalTok{)}
\NormalTok{plt.ylabel(}\StringTok{"Measured signal [a.u.]"}\NormalTok{, fontsize}\OperatorTok{=}\DecValTok{15}\NormalTok{)}
\NormalTok{plt.show()}
\end{Highlighting}
\end{Shaded}

\begin{verbatim}
[ 0.22467697 -0.01074188  0.38619543  3.1538365 ]
\end{verbatim}

\includegraphics{e8ac546faebf5cdb69d184f66d927b92685c4251.png}

\hypertarget{096f5118-d5f5-45c5-8841-261af97ab9b7}{}
\begin{Shaded}
\begin{Highlighting}[]
\NormalTok{pi\_amp }\OperatorTok{=} \BuiltInTok{abs}\NormalTok{(drive\_period }\OperatorTok{/} \DecValTok{2}\NormalTok{)}
\BuiltInTok{print}\NormalTok{(}\SpecialStringTok{f"Pi Amplitude = }\SpecialCharTok{\{}\NormalTok{pi\_amp}\SpecialCharTok{\}}\SpecialStringTok{"}\NormalTok{)}
\end{Highlighting}
\end{Shaded}

\begin{verbatim}
Pi Amplitude = 0.19309771515336357
\end{verbatim}

\hypertarget{8ecf2349-58a1-4390-a8cb-0616b221c41f}{}
\hypertarget{our-pi-pulse}{%
\subsubsection{\texorpdfstring{Our \(\pi\)
pulse!}{Our \textbackslash pi pulse!}}\label{our-pi-pulse}}

Let's define our pulse, with the amplitude we just found, so we can use
it in later experiments.

\hypertarget{ef7d0882-174c-4462-8d00-efb23910c081}{}
\begin{Shaded}
\begin{Highlighting}[]
\ControlFlowTok{with}\NormalTok{ pulse.build(backend) }\ImportTok{as}\NormalTok{ pi\_pulse\_0\_1:}
\NormalTok{    drive\_duration }\OperatorTok{=}\NormalTok{ get\_closest\_multiple\_of\_16(pulse.seconds\_to\_samples(drive\_duration\_sec))}
\NormalTok{    drive\_sigma }\OperatorTok{=}\NormalTok{ pulse.seconds\_to\_samples(drive\_sigma\_sec)}
\NormalTok{    drive\_chan }\OperatorTok{=}\NormalTok{ pulse.drive\_channel(qubit)}
\NormalTok{    pulse.play(pulse.Gaussian(duration}\OperatorTok{=}\NormalTok{drive\_duration,}
\NormalTok{                              amp}\OperatorTok{=}\NormalTok{pi\_amp,}
\NormalTok{                              sigma}\OperatorTok{=}\NormalTok{drive\_sigma,}
\NormalTok{                              name}\OperatorTok{=}\StringTok{\textquotesingle{}pi\_pulse\textquotesingle{}}\NormalTok{), drive\_chan)}
\NormalTok{pi\_pulse\_0\_1.draw(backend}\OperatorTok{=}\NormalTok{backend)}
\end{Highlighting}
\end{Shaded}

\includegraphics{f37d255ba5902c4df2078518f3a41ad1d4cceaa2.png}

\hypertarget{3e360688-2184-4587-8098-cd39b505140b}{}
\hypertarget{initial-freq-1-2}{%
\section{Initial Freq 1-\textgreater2}\label{initial-freq-1-2}}

\hypertarget{483582f2-0e4c-4577-9bf4-c2e9625bbe27}{}
\begin{Shaded}
\begin{Highlighting}[]
\CommentTok{\# We will sweep 40 MHz around the estimated frequency}
\NormalTok{frequency\_span\_Hz }\OperatorTok{=} \DecValTok{40} \OperatorTok{*}\NormalTok{ MHz}
\CommentTok{\# in steps of 1 MHz.}
\NormalTok{frequency\_step\_Hz }\OperatorTok{=} \DecValTok{1} \OperatorTok{*}\NormalTok{ MHz}
\NormalTok{a}\OperatorTok{={-}}\FloatTok{0.35}
\CommentTok{\# We will sweep 20 MHz above and 20 MHz below the estimated frequency}
\NormalTok{frequency\_min }\OperatorTok{=}\NormalTok{ (center\_frequency\_Hz }\OperatorTok{{-}}\NormalTok{ frequency\_span\_Hz }\OperatorTok{/} \DecValTok{2}\NormalTok{)}\OperatorTok{+}\NormalTok{a}\OperatorTok{*}\NormalTok{GHz}
\NormalTok{frequency\_max }\OperatorTok{=}\NormalTok{ (center\_frequency\_Hz }\OperatorTok{+}\NormalTok{ frequency\_span\_Hz }\OperatorTok{/} \DecValTok{2}\NormalTok{)}\OperatorTok{+}\NormalTok{a}\OperatorTok{*}\NormalTok{GHz}
\CommentTok{\# Construct an np array of the frequencies for our experiment}
\NormalTok{frequencies\_GHz }\OperatorTok{=}\NormalTok{ np.arange((frequency\_min }\OperatorTok{/}\NormalTok{ GHz), }
\NormalTok{                            (frequency\_max }\OperatorTok{/}\NormalTok{ GHz), }
\NormalTok{                            frequency\_step\_Hz }\OperatorTok{/}\NormalTok{ GHz)}

\BuiltInTok{print}\NormalTok{(}\SpecialStringTok{f"The sweep will go from }\SpecialCharTok{\{}\NormalTok{frequency\_min }\OperatorTok{/}\NormalTok{ GHz}\SpecialCharTok{\}}\SpecialStringTok{ GHz to }\SpecialCharTok{\{}\NormalTok{frequency\_max }\OperatorTok{/}\NormalTok{ GHz}\SpecialCharTok{\}}\SpecialStringTok{ GHz }\CharTok{\textbackslash{}}
\SpecialStringTok{in steps of }\SpecialCharTok{\{}\NormalTok{frequency\_step\_Hz }\OperatorTok{/}\NormalTok{ MHz}\SpecialCharTok{\}}\SpecialStringTok{ MHz."}\NormalTok{)}
\BuiltInTok{print}\NormalTok{(}\BuiltInTok{len}\NormalTok{(frequencies\_GHz))}
\end{Highlighting}
\end{Shaded}

\begin{verbatim}
The sweep will go from 4.601680543795522 GHz to 4.641680543795522 GHz in steps of 1.0 MHz.
41
\end{verbatim}

\hypertarget{d26b1955-4c0c-4188-93a8-186baf4ffac1}{}
\begin{Shaded}
\begin{Highlighting}[]
\ImportTok{from}\NormalTok{ qiskit }\ImportTok{import}\NormalTok{ pulse                  }\CommentTok{\# This is where we access all of our Pulse features!}
\ImportTok{from}\NormalTok{ qiskit.circuit }\ImportTok{import}\NormalTok{ Parameter      }\CommentTok{\# This is Parameter Class for variable parameters.}

\CommentTok{\# Drive pulse parameters (us = microseconds)}
\NormalTok{drive\_sigma\_sec }\OperatorTok{=} \FloatTok{0.015} \OperatorTok{*}\NormalTok{ us                      }\CommentTok{\# This determines the actual width of the gaussian}
\NormalTok{drive\_duration\_sec }\OperatorTok{=}\NormalTok{ drive\_sigma\_sec }\OperatorTok{*} \DecValTok{8}                \CommentTok{\# This is a truncating parameter, because gaussians don\textquotesingle{}t have }
                                                        \CommentTok{\# a natural finite length}
\NormalTok{drive\_amp }\OperatorTok{=} \FloatTok{0.7}

\NormalTok{frequencies\_Hz }\OperatorTok{=}\NormalTok{ frequencies\_GHz}\OperatorTok{*}\NormalTok{GHz}
\CommentTok{\# Create the base schedule}
\CommentTok{\# Start with drive pulse acting on the drive channel}
\NormalTok{freq }\OperatorTok{=}\NormalTok{ Parameter(}\StringTok{\textquotesingle{}freq\textquotesingle{}}\NormalTok{)}
\ControlFlowTok{with}\NormalTok{ pulse.build(backend}\OperatorTok{=}\NormalTok{backend, default\_alignment}\OperatorTok{=}\StringTok{\textquotesingle{}sequential\textquotesingle{}}\NormalTok{, name}\OperatorTok{=}\StringTok{\textquotesingle{}Frequency sweep\textquotesingle{}}\NormalTok{) }\ImportTok{as}\NormalTok{ sweep\_sched:}
\NormalTok{    drive\_duration }\OperatorTok{=}\NormalTok{ get\_closest\_multiple\_of\_16(pulse.seconds\_to\_samples(drive\_duration\_sec))}
\NormalTok{    drive\_sigma }\OperatorTok{=}\NormalTok{ pulse.seconds\_to\_samples(drive\_sigma\_sec)}
\NormalTok{    drive\_chan }\OperatorTok{=}\NormalTok{ pulse.drive\_channel(qubit)}
\NormalTok{    pulse.set\_frequency(rough\_qubit\_frequency, drive\_chan)}
\NormalTok{    pulse.call(pi\_pulse\_0\_1)}
\NormalTok{    pulse.set\_frequency(freq, drive\_chan)}
    \CommentTok{\# Drive pulse samples}
    \CommentTok{\#pulse.play(pulse.library.Constant(drive\_duration, drive\_amp), drive\_chan)}
\NormalTok{    pulse.play(pulse.Gaussian(duration}\OperatorTok{=}\NormalTok{drive\_duration,sigma}\OperatorTok{=}\NormalTok{drive\_sigma,amp}\OperatorTok{=}\NormalTok{drive\_amp,name}\OperatorTok{=}\StringTok{\textquotesingle{}freq\_sweep\_excitation\_pulse\textquotesingle{}}\NormalTok{), drive\_chan)}
    \CommentTok{\# Define our measurement pulse}
\NormalTok{    pulse.measure(qubits}\OperatorTok{=}\NormalTok{[qubit], registers}\OperatorTok{=}\NormalTok{[pulse.MemorySlot(mem\_slot)])}
        

\CommentTok{\# Create the frequency settings for the sweep (MUST BE IN HZ)}
\NormalTok{schedules }\OperatorTok{=}\NormalTok{ [sweep\_sched.assign\_parameters(\{freq: f\}, inplace}\OperatorTok{=}\VariableTok{False}\NormalTok{) }\ControlFlowTok{for}\NormalTok{ f }\KeywordTok{in}\NormalTok{ frequencies\_Hz]}
\end{Highlighting}
\end{Shaded}

\hypertarget{1ca39cea-2a80-4b33-b814-e755e74e77e8}{}
\begin{Shaded}
\begin{Highlighting}[]
\NormalTok{schedules[}\DecValTok{0}\NormalTok{].draw(backend}\OperatorTok{=}\NormalTok{backend)}
\NormalTok{sweep\_sched.draw(backend}\OperatorTok{=}\NormalTok{backend)}
\end{Highlighting}
\end{Shaded}

\includegraphics{397366dd59023f374551c1de3a09d307473fffd0.png}

\hypertarget{87f87c2f-8895-4a71-afe0-85b668282e0e}{}
\begin{Shaded}
\begin{Highlighting}[]
\NormalTok{num\_shots\_per\_frequency }\OperatorTok{=} \DecValTok{1024}
\NormalTok{job }\OperatorTok{=}\NormalTok{ backend.run(schedules, }
\NormalTok{                  meas\_level}\OperatorTok{=}\DecValTok{1}\NormalTok{, }
\NormalTok{                  meas\_return}\OperatorTok{=}\StringTok{\textquotesingle{}avg\textquotesingle{}}\NormalTok{, }
\NormalTok{                  shots}\OperatorTok{=}\NormalTok{num\_shots\_per\_frequency)}
\NormalTok{job\_monitor(job)}
\end{Highlighting}
\end{Shaded}

\begin{verbatim}
Job Status: job has successfully run
\end{verbatim}

\hypertarget{10af25c8-7496-4d88-98a9-2c42457df3ca}{}
\begin{Shaded}
\begin{Highlighting}[]
\NormalTok{frequency\_sweep\_results}\OperatorTok{=}\NormalTok{ job.result(timeout}\OperatorTok{=}\DecValTok{120}\NormalTok{) }\CommentTok{\# timeout parameter set to 120 second}
\end{Highlighting}
\end{Shaded}

\hypertarget{6c76be86-89b7-43c2-b70c-c8a243c2cf2c}{}
\begin{Shaded}
\begin{Highlighting}[]
\ImportTok{import}\NormalTok{ matplotlib.pyplot }\ImportTok{as}\NormalTok{ plt}

\NormalTok{sweep\_values }\OperatorTok{=}\NormalTok{ []}
\ControlFlowTok{for}\NormalTok{ i }\KeywordTok{in} \BuiltInTok{range}\NormalTok{(}\BuiltInTok{len}\NormalTok{(frequency\_sweep\_results.results)):}
    \CommentTok{\# Get the results from the ith experiment}
\NormalTok{    res }\OperatorTok{=}\NormalTok{ frequency\_sweep\_results.get\_memory(i)}\OperatorTok{*}\NormalTok{scale\_factor}
    \CommentTok{\# Get the results for \textasciigrave{}qubit\textasciigrave{} from this experiment}
\NormalTok{    sweep\_values.append(res[qubit])}
    \CommentTok{\#if (i\%4==0): print((frequency\_sweep\_results.get\_memory(i)*scale\_factor)[0])}

\NormalTok{plt.scatter(frequencies\_GHz, np.real(sweep\_values), color}\OperatorTok{=}\StringTok{\textquotesingle{}black\textquotesingle{}}\NormalTok{) }\CommentTok{\# plot real part of sweep values}
\NormalTok{plt.xlim([}\BuiltInTok{min}\NormalTok{(frequencies\_GHz), }\BuiltInTok{max}\NormalTok{(frequencies\_GHz)])}
\NormalTok{plt.xlabel(}\StringTok{"Frequency [GHz]"}\NormalTok{)}
\NormalTok{plt.ylabel(}\StringTok{"Measured signal [a.u.]"}\NormalTok{)}
\NormalTok{plt.show()}
\end{Highlighting}
\end{Shaded}

\includegraphics{90875f143deae12f4f0ba35d7bcacc882d428e38.png}

\hypertarget{e02e3234-6713-4669-af17-075c85c2cca4}{}
\begin{Shaded}
\begin{Highlighting}[]
\NormalTok{fit\_params, y\_fit }\OperatorTok{=}\NormalTok{ fit\_function(frequencies\_GHz,}
\NormalTok{                                 np.real(sweep\_values), }
                                 \KeywordTok{lambda}\NormalTok{ x, A, q\_freq, B, C: ((A}\OperatorTok{/}\NormalTok{np.pi)}\OperatorTok{*}\NormalTok{(B}\OperatorTok{/}\NormalTok{ ((x}\OperatorTok{{-}}\NormalTok{q\_freq)}\OperatorTok{**}\DecValTok{2}\OperatorTok{+}\NormalTok{B}\OperatorTok{**}\DecValTok{2}\NormalTok{))}\OperatorTok{+}\NormalTok{C),}
\NormalTok{                                 [}\FloatTok{0.003}\NormalTok{, }\FloatTok{4.62}\NormalTok{, }\FloatTok{0.003}\NormalTok{,}\OperatorTok{{-}}\FloatTok{1.2}\NormalTok{] }\CommentTok{\# initial parameters for curve\_fit}
\NormalTok{                                )}
\NormalTok{plt.scatter(frequencies\_GHz, np.real(sweep\_values), color}\OperatorTok{=}\StringTok{\textquotesingle{}black\textquotesingle{}}\NormalTok{)}
\NormalTok{plt.plot(frequencies\_GHz, y\_fit, color}\OperatorTok{=}\StringTok{\textquotesingle{}red\textquotesingle{}}\NormalTok{)}
\NormalTok{plt.xlim([}\BuiltInTok{min}\NormalTok{(frequencies\_GHz), }\BuiltInTok{max}\NormalTok{(frequencies\_GHz)])}

\NormalTok{plt.xlabel(}\StringTok{"Frequency [GHz]"}\NormalTok{)}
\NormalTok{plt.ylabel(}\StringTok{"Measured Signal [a.u.]"}\NormalTok{)}
\NormalTok{plt.show()}
\end{Highlighting}
\end{Shaded}

\includegraphics{dfc6a7c1ed20c2ce87f5840cd7944b43a5e1e014.png}

\hypertarget{b4801155-c44f-453e-b968-d0fb205685ca}{}
\begin{Shaded}
\begin{Highlighting}[]
\NormalTok{A, rough\_qubit\_frequency\_2, B, C }\OperatorTok{=}\NormalTok{ fit\_params}
\NormalTok{rough\_qubit\_frequency\_2 }\OperatorTok{=}\NormalTok{ rough\_qubit\_frequency\_2}\OperatorTok{*}\NormalTok{GHz }\CommentTok{\# make sure qubit freq is in Hz}
\BuiltInTok{print}\NormalTok{(}\SpecialStringTok{f"We\textquotesingle{}ve updated our qubit frequency estimate from "}
      \SpecialStringTok{f"}\SpecialCharTok{\{}\BuiltInTok{round}\NormalTok{(backend\_defaults.qubit\_freq\_est[qubit] }\OperatorTok{/}\NormalTok{ GHz, }\DecValTok{5}\NormalTok{)}\SpecialCharTok{\}}\SpecialStringTok{ GHz to }\SpecialCharTok{\{}\BuiltInTok{round}\NormalTok{(rough\_qubit\_frequency\_2}\OperatorTok{/}\NormalTok{GHz, }\DecValTok{5}\NormalTok{)}\SpecialCharTok{\}}\SpecialStringTok{ GHz."}\NormalTok{)}
\CommentTok{\#print(fit\_params)}
\end{Highlighting}
\end{Shaded}

\begin{verbatim}
We've updated our qubit frequency estimate from 4.97168 GHz to 4.62324 GHz.
\end{verbatim}

\hypertarget{6e765aa4-4d58-4e8b-a3a3-0cc03a02d309}{}
\hypertarget{using-rabi-to-calibrate-the-pi-pulse-for-1---2-}{%
\section{\texorpdfstring{Using Rabi to Calibrate the \(\pi\) Pulse for
\textbar1\textgreater{} -\textgreater{} \textbar2\textgreater{}
}{Using Rabi to Calibrate the \textbackslash pi Pulse for \textbar1\textgreater{} -\textgreater{} \textbar2\textgreater{} }}\label{using-rabi-to-calibrate-the-pi-pulse-for-1---2-}}

\hypertarget{calibrating-pi-pulses-using-a-rabi-experiment-}{%
\subsubsection{\texorpdfstring{Calibrating \(\pi\) Pulses using a Rabi
Experiment
}{Calibrating \textbackslash pi Pulses using a Rabi Experiment }}\label{calibrating-pi-pulses-using-a-rabi-experiment-}}

Once we know the frequency of our qubit, the next step is to determine
the strength of a \(\pi\) pulse. Strictly speaking of the qubit as a
two-level system, a \(\pi\) pulse is one that takes the qubit from
\(\vert0\rangle\) to \(\vert1\rangle\), and vice versa. This is also
called the \(X\) or \(X180\) gate, or bit-flip operator. We already know
the microwave frequency needed to drive this transition from the
previous frequency sweep experiment, and we now seek the amplitude
needed to achieve a \(\pi\) rotation from \(\vert0\rangle\) to
\(\vert1\rangle\). The desired rotation is shown on the Bloch sphere in
the figure below -\/- you can see that the \(\pi\) pulse gets its name
from the angle it sweeps over on a Bloch sphere.

\hypertarget{d0409108-e9b8-4218-9f97-8534a142740b}{}
\begin{Shaded}
\begin{Highlighting}[]
\CommentTok{\# Build the Rabi experiments:}
\CommentTok{\#    A drive pulse at the qubit frequency, followed by a measurement,}
\CommentTok{\#    where we vary the drive amplitude each time.}

\NormalTok{drive\_amp }\OperatorTok{=}\NormalTok{ Parameter(}\StringTok{\textquotesingle{}drive\_amp\textquotesingle{}}\NormalTok{)}
\ControlFlowTok{with}\NormalTok{ pulse.build(backend}\OperatorTok{=}\NormalTok{backend, default\_alignment}\OperatorTok{=}\StringTok{\textquotesingle{}sequential\textquotesingle{}}\NormalTok{, name}\OperatorTok{=}\StringTok{\textquotesingle{}Rabi Experiment\textquotesingle{}}\NormalTok{) }\ImportTok{as}\NormalTok{ rabi\_sched\_2:}
\NormalTok{    drive\_duration }\OperatorTok{=}\NormalTok{ get\_closest\_multiple\_of\_16(pulse.seconds\_to\_samples(drive\_duration\_sec}\OperatorTok{*}\DecValTok{2}\NormalTok{))}
\NormalTok{    drive\_sigma }\OperatorTok{=}\NormalTok{ pulse.seconds\_to\_samples(drive\_sigma\_sec}\OperatorTok{*}\DecValTok{2}\NormalTok{)}
\NormalTok{    drive\_chan }\OperatorTok{=}\NormalTok{ pulse.drive\_channel(qubit)}
\NormalTok{    pulse.set\_frequency(rough\_qubit\_frequency, drive\_chan)}
\NormalTok{    pulse.call(pi\_pulse\_0\_1)}
\NormalTok{    pulse.set\_frequency(rough\_qubit\_frequency\_2, drive\_chan)}
\NormalTok{    pulse.play(pulse.Gaussian(duration}\OperatorTok{=}\NormalTok{drive\_duration,amp}\OperatorTok{=}\NormalTok{drive\_amp,sigma}\OperatorTok{=}\NormalTok{drive\_sigma,name}\OperatorTok{=}\StringTok{\textquotesingle{}Rabi Pulse\textquotesingle{}}\NormalTok{), drive\_chan)}
\NormalTok{    pulse.measure(qubits}\OperatorTok{=}\NormalTok{[qubit], registers}\OperatorTok{=}\NormalTok{[pulse.MemorySlot(mem\_slot)])}

\NormalTok{rabi\_schedules\_2 }\OperatorTok{=}\NormalTok{ [rabi\_sched\_2.assign\_parameters(\{drive\_amp: a\}, inplace}\OperatorTok{=}\VariableTok{False}\NormalTok{) }\ControlFlowTok{for}\NormalTok{ a }\KeywordTok{in}\NormalTok{ drive\_amps]}
\end{Highlighting}
\end{Shaded}

\hypertarget{7d7c966d-076a-4b3b-b003-73ec0f6584e8}{}
\begin{Shaded}
\begin{Highlighting}[]
\NormalTok{rabi\_schedules\_2[}\DecValTok{2}\NormalTok{].draw(backend}\OperatorTok{=}\NormalTok{backend)}
\NormalTok{rabi\_sched\_2.draw(backend}\OperatorTok{=}\NormalTok{backend)}
\end{Highlighting}
\end{Shaded}

\includegraphics{ad3df40ed24ac43a7f7f53c316d582e3277cdd40.png}

\hypertarget{2e6ca967-ac35-4416-81e8-4f05fe167c81}{}
\begin{Shaded}
\begin{Highlighting}[]
\NormalTok{num\_shots\_per\_point }\OperatorTok{=} \DecValTok{2000}

\NormalTok{job }\OperatorTok{=}\NormalTok{ backend.run(rabi\_schedules\_2, }
\NormalTok{                  meas\_level}\OperatorTok{=}\DecValTok{1}\NormalTok{, }
\NormalTok{                  meas\_return}\OperatorTok{=}\StringTok{\textquotesingle{}avg\textquotesingle{}}\NormalTok{, }
\NormalTok{                  shots}\OperatorTok{=}\NormalTok{num\_shots\_per\_point)}

\NormalTok{job\_monitor(job)}
\end{Highlighting}
\end{Shaded}

\begin{verbatim}
Job Status: job has successfully run
\end{verbatim}

\hypertarget{01730ef4-fcfe-4e9d-ba3b-84b3ed84091a}{}
\begin{Shaded}
\begin{Highlighting}[]
\NormalTok{rabi\_results }\OperatorTok{=}\NormalTok{ job.result(timeout}\OperatorTok{=}\DecValTok{120}\NormalTok{)}
\end{Highlighting}
\end{Shaded}

\hypertarget{13271f34-bf07-435c-a84f-e7805adfb0a7}{}
\begin{Shaded}
\begin{Highlighting}[]
\NormalTok{rabi\_values }\OperatorTok{=}\NormalTok{ []}
\ControlFlowTok{for}\NormalTok{ i }\KeywordTok{in} \BuiltInTok{range}\NormalTok{(num\_rabi\_points):}
    \CommentTok{\# Get the results for \textasciigrave{}qubit\textasciigrave{} from the ith experiment}
\NormalTok{    rabi\_values.append(rabi\_results.get\_memory(i)[qubit] }\OperatorTok{*}\NormalTok{ scale\_factor)}

\NormalTok{rabi\_values }\OperatorTok{=}\NormalTok{ np.real(baseline\_remove(rabi\_values))}

\NormalTok{plt.xlabel(}\StringTok{"Drive amp [a.u.]"}\NormalTok{)}
\NormalTok{plt.ylabel(}\StringTok{"Measured signal [a.u.]"}\NormalTok{)}
\NormalTok{plt.scatter(drive\_amps, rabi\_values, color}\OperatorTok{=}\StringTok{\textquotesingle{}black\textquotesingle{}}\NormalTok{) }\CommentTok{\# plot real part of Rabi values}
\NormalTok{plt.show()}
\end{Highlighting}
\end{Shaded}

\includegraphics{5943de6d64ba39d9b53f54dc2c12d8af4799c881.png}

\hypertarget{89ec61b4-4220-41fc-af6e-59f59f784009}{}
\begin{Shaded}
\begin{Highlighting}[]
\NormalTok{fit\_params, y\_fit }\OperatorTok{=}\NormalTok{ fit\_function(drive\_amps,}
\NormalTok{                                 rabi\_values, }
                                 \KeywordTok{lambda}\NormalTok{ x, A, B, drive\_period, phi: (A}\OperatorTok{*}\NormalTok{np.cos(}\DecValTok{2}\OperatorTok{*}\NormalTok{np.pi}\OperatorTok{*}\NormalTok{x}\OperatorTok{/}\NormalTok{drive\_period }\OperatorTok{{-}}\NormalTok{ phi) }\OperatorTok{+}\NormalTok{ B),}
\NormalTok{                                 [}\FloatTok{0.2}\NormalTok{, }\OperatorTok{{-}}\FloatTok{0.2}\NormalTok{, }\FloatTok{0.75}\NormalTok{, np.pi}\OperatorTok{/}\DecValTok{2}\NormalTok{])}

\NormalTok{plt.scatter(drive\_amps, rabi\_values, color}\OperatorTok{=}\StringTok{\textquotesingle{}black\textquotesingle{}}\NormalTok{)}
\NormalTok{plt.plot(drive\_amps, y\_fit, color}\OperatorTok{=}\StringTok{\textquotesingle{}red\textquotesingle{}}\NormalTok{)}
\BuiltInTok{print}\NormalTok{(fit\_params)}
\NormalTok{drive\_period\_2 }\OperatorTok{=}\NormalTok{ fit\_params[}\DecValTok{2}\NormalTok{] }\CommentTok{\# get period of rabi oscillation}

\NormalTok{plt.axvline(drive\_period\_2}\OperatorTok{/}\DecValTok{2}\NormalTok{, color}\OperatorTok{=}\StringTok{\textquotesingle{}red\textquotesingle{}}\NormalTok{, linestyle}\OperatorTok{=}\StringTok{\textquotesingle{}{-}{-}\textquotesingle{}}\NormalTok{)}
\NormalTok{plt.axvline(drive\_period\_2, color}\OperatorTok{=}\StringTok{\textquotesingle{}red\textquotesingle{}}\NormalTok{, linestyle}\OperatorTok{=}\StringTok{\textquotesingle{}{-}{-}\textquotesingle{}}\NormalTok{)}
\NormalTok{plt.annotate(}\StringTok{""}\NormalTok{, xy}\OperatorTok{=}\NormalTok{(drive\_period\_2, }\DecValTok{0}\NormalTok{), xytext}\OperatorTok{=}\NormalTok{(drive\_period}\OperatorTok{/}\DecValTok{2}\NormalTok{,}\DecValTok{0}\NormalTok{), arrowprops}\OperatorTok{=}\BuiltInTok{dict}\NormalTok{(arrowstyle}\OperatorTok{=}\StringTok{"\textless{}{-}\textgreater{}"}\NormalTok{, color}\OperatorTok{=}\StringTok{\textquotesingle{}red\textquotesingle{}}\NormalTok{))}
\NormalTok{plt.annotate(}\StringTok{"$\textbackslash{}pi$"}\NormalTok{, xy}\OperatorTok{=}\NormalTok{(drive\_period\_2}\OperatorTok{/}\DecValTok{2}\OperatorTok{{-}}\FloatTok{0.03}\NormalTok{, }\FloatTok{0.1}\NormalTok{), color}\OperatorTok{=}\StringTok{\textquotesingle{}red\textquotesingle{}}\NormalTok{)}

\NormalTok{plt.xlabel(}\StringTok{"Drive amp [a.u.]"}\NormalTok{, fontsize}\OperatorTok{=}\DecValTok{15}\NormalTok{)}
\NormalTok{plt.ylabel(}\StringTok{"Measured signal [a.u.]"}\NormalTok{, fontsize}\OperatorTok{=}\DecValTok{15}\NormalTok{)}
\NormalTok{plt.show()}

\NormalTok{pi\_amp\_2 }\OperatorTok{=} \BuiltInTok{abs}\NormalTok{(drive\_period\_2 }\OperatorTok{/} \DecValTok{2}\NormalTok{)}
\BuiltInTok{print}\NormalTok{(}\SpecialStringTok{f"Pi Amplitude = }\SpecialCharTok{\{}\NormalTok{pi\_amp\_2}\SpecialCharTok{\}}\SpecialStringTok{"}\NormalTok{)}
\end{Highlighting}
\end{Shaded}

\begin{verbatim}
[0.04806702 0.00375785 0.73842967 3.10003536]
\end{verbatim}

\includegraphics{bc69d0822f4bf123bade05abc5c1e4a6402ba06b.png}

\begin{verbatim}
Pi Amplitude = 0.3692148371978982
\end{verbatim}

\hypertarget{2690bc89-ed3f-4803-a628-7ca3a5a45569}{}
\hypertarget{our-1---2-pi-pulse}{%
\subsubsection{\texorpdfstring{Our \textbar1\textgreater{}
-\textgreater{} \textbar2\textgreater{} \(\pi\)
pulse!}{Our \textbar1\textgreater{} -\textgreater{} \textbar2\textgreater{} \textbackslash pi pulse!}}\label{our-1---2-pi-pulse}}

Let's define our pulse, with the amplitude we just found, so we can use
it in later experiments.

\hypertarget{152b6d6c-24f1-43df-be9b-05d093007719}{}
\begin{Shaded}
\begin{Highlighting}[]
\ControlFlowTok{with}\NormalTok{ pulse.build(backend) }\ImportTok{as}\NormalTok{ pi\_pulse\_0\_2:}
\NormalTok{    drive\_duration }\OperatorTok{=}\NormalTok{ get\_closest\_multiple\_of\_16(pulse.seconds\_to\_samples(drive\_duration\_sec}\OperatorTok{*}\DecValTok{2}\NormalTok{))}
\NormalTok{    drive\_sigma }\OperatorTok{=}\NormalTok{ pulse.seconds\_to\_samples(drive\_sigma\_sec}\OperatorTok{*}\DecValTok{2}\NormalTok{)}
\NormalTok{    drive\_chan }\OperatorTok{=}\NormalTok{ pulse.drive\_channel(qubit)}
\NormalTok{    pulse.play(pulse.Gaussian(duration}\OperatorTok{=}\NormalTok{drive\_duration,}
\NormalTok{                              amp}\OperatorTok{=}\NormalTok{pi\_amp\_2,}
\NormalTok{                              sigma}\OperatorTok{=}\NormalTok{drive\_sigma,}
\NormalTok{                              name}\OperatorTok{=}\StringTok{\textquotesingle{}pi\_pulse\_1{-}\textgreater{}2\textquotesingle{}}\NormalTok{), drive\_chan)}
\NormalTok{pi\_pulse\_0\_2.draw(backend}\OperatorTok{=}\NormalTok{backend)}
\end{Highlighting}
\end{Shaded}

\includegraphics{702452c971506411334f6c003581601693a592b2.png}

\hypertarget{8519faaf-8459-4741-840a-c4a96a5ccc99}{}
\hypertarget{determining-0-vs-1-vs-2-}{%
\section{\texorpdfstring{Determining 0 vs 1 vs 2
}{Determining 0 vs 1 vs 2 }}\label{determining-0-vs-1-vs-2-}}

\leavevmode\vadjust pre{\hypertarget{91182d96-f00a-49a2-bff3-e2826a62b8c0}{}}%
Once our \(\pi\) pulses have been calibrated, we can now create the
state \(\vert1\rangle\) with good probability. We can use this to find
out what the states \(\vert0\rangle\) and \(\vert1\rangle\) look like in
our measurements, by repeatedly preparing them and plotting the measured
signal. This is what we use to build a discriminator, which is simply a
function which takes a measured and kerneled complex value
(\texttt{meas\_level=1}) and classifies it as a 0 or a 1
(\texttt{meas\_level=2}).

\hypertarget{0c33f2ee-a9cb-4257-9c88-c9f74edbc15b}{}
\begin{Shaded}
\begin{Highlighting}[]
\CommentTok{\# Create two schedules}

\CommentTok{\# Ground state schedule}
\ControlFlowTok{with}\NormalTok{ pulse.build(backend}\OperatorTok{=}\NormalTok{backend, default\_alignment}\OperatorTok{=}\StringTok{\textquotesingle{}sequential\textquotesingle{}}\NormalTok{, name}\OperatorTok{=}\StringTok{\textquotesingle{}ground state\textquotesingle{}}\NormalTok{) }\ImportTok{as}\NormalTok{ gnd\_schedule:}
\NormalTok{    drive\_chan }\OperatorTok{=}\NormalTok{ pulse.drive\_channel(qubit)}
\NormalTok{    pulse.set\_frequency(rough\_qubit\_frequency, drive\_chan)}
\NormalTok{    pulse.measure(qubits}\OperatorTok{=}\NormalTok{[qubit], registers}\OperatorTok{=}\NormalTok{[pulse.MemorySlot(mem\_slot)])}


\CommentTok{\# Excited state schedule}
\ControlFlowTok{with}\NormalTok{ pulse.build(backend}\OperatorTok{=}\NormalTok{backend, default\_alignment}\OperatorTok{=}\StringTok{\textquotesingle{}sequential\textquotesingle{}}\NormalTok{, name}\OperatorTok{=}\StringTok{\textquotesingle{}excited state\textquotesingle{}}\NormalTok{) }\ImportTok{as}\NormalTok{ exc\_1\_schedule:}
\NormalTok{    drive\_chan }\OperatorTok{=}\NormalTok{ pulse.drive\_channel(qubit)}
\NormalTok{    pulse.set\_frequency(rough\_qubit\_frequency, drive\_chan)}
\NormalTok{    pulse.call(pi\_pulse\_0\_1)}
\NormalTok{    pulse.measure(qubits}\OperatorTok{=}\NormalTok{[qubit], registers}\OperatorTok{=}\NormalTok{[pulse.MemorySlot(mem\_slot)])}
    
\ControlFlowTok{with}\NormalTok{ pulse.build(backend}\OperatorTok{=}\NormalTok{backend, default\_alignment}\OperatorTok{=}\StringTok{\textquotesingle{}sequential\textquotesingle{}}\NormalTok{, name}\OperatorTok{=}\StringTok{\textquotesingle{}excited state\textquotesingle{}}\NormalTok{) }\ImportTok{as}\NormalTok{ exc\_2\_schedule:}
\NormalTok{    drive\_chan }\OperatorTok{=}\NormalTok{ pulse.drive\_channel(qubit)}
\NormalTok{    pulse.set\_frequency(rough\_qubit\_frequency, drive\_chan)}
\NormalTok{    pulse.call(pi\_pulse\_0\_1)}
\NormalTok{    pulse.set\_frequency(rough\_qubit\_frequency\_2, drive\_chan)}
\NormalTok{    pulse.call(pi\_pulse\_0\_2)}
\NormalTok{    pulse.measure(qubits}\OperatorTok{=}\NormalTok{[qubit], registers}\OperatorTok{=}\NormalTok{[pulse.MemorySlot(mem\_slot)])}
\end{Highlighting}
\end{Shaded}

\hypertarget{4aac5b01-78a6-454c-ac14-15c2fd9c1bf8}{}
\begin{Shaded}
\begin{Highlighting}[]
\NormalTok{gnd\_schedule.draw(backend}\OperatorTok{=}\NormalTok{backend)}
\end{Highlighting}
\end{Shaded}

\includegraphics{4d276a8e79ba1370100e39864a937723116a6ed1.png}

\hypertarget{13189420-2bd5-4ebb-a2b1-eecda68f8410}{}
\begin{Shaded}
\begin{Highlighting}[]
\NormalTok{exc\_1\_schedule.draw(backend}\OperatorTok{=}\NormalTok{backend)}
\end{Highlighting}
\end{Shaded}

\includegraphics{11037ad312f719040f866c2c369139da68591ed4.png}

\hypertarget{3d8a05ab-3dac-46b0-ad9a-f528a4c7ab84}{}
\begin{Shaded}
\begin{Highlighting}[]
\NormalTok{exc\_2\_schedule.draw(backend}\OperatorTok{=}\NormalTok{backend)}
\end{Highlighting}
\end{Shaded}

\includegraphics{35dcf5a5e7284e1ac0be9ebafa1ee13df297eb91.png}

\leavevmode\vadjust pre{\hypertarget{7256ce34-2dba-4186-8016-0416ac19c7dd}{}}%
We assemble the ground and excited state preparation schedules. Each of
these will run \texttt{num\_shots} times. We choose
\texttt{meas\_level=1} this time, because we do not want the results
already classified for us as \(|0\rangle\) or \(|1\rangle\). Instead, we
want kerneled data: raw acquired data that has gone through a kernel
function to yield a single complex value for each shot. (You can think
of a kernel as a dot product applied to the raw measurement data.)

\hypertarget{9ddcb126-e885-411a-99df-eb1e65cfb8b8}{}
\begin{Shaded}
\begin{Highlighting}[]
\CommentTok{\# Execution settings}
\NormalTok{num\_shots }\OperatorTok{=} \DecValTok{1024}

\NormalTok{job }\OperatorTok{=}\NormalTok{ backend.run([gnd\_schedule, exc\_1\_schedule, exc\_2\_schedule], }
\NormalTok{                  meas\_level}\OperatorTok{=}\DecValTok{1}\NormalTok{, }
\NormalTok{                  meas\_return}\OperatorTok{=}\StringTok{\textquotesingle{}single\textquotesingle{}}\NormalTok{, }
\NormalTok{                  shots}\OperatorTok{=}\NormalTok{num\_shots)}

\NormalTok{job\_monitor(job)}
\end{Highlighting}
\end{Shaded}

\begin{verbatim}
Job Status: job has successfully run
\end{verbatim}

\hypertarget{823c030a-3335-4818-986b-9159523b885d}{}
\begin{Shaded}
\begin{Highlighting}[]
\NormalTok{gnd\_exc\_results }\OperatorTok{=}\NormalTok{ job.result(timeout}\OperatorTok{=}\DecValTok{120}\NormalTok{)}
\end{Highlighting}
\end{Shaded}

\leavevmode\vadjust pre{\hypertarget{1bc4f47c-6a9a-4380-b9b3-8555b22121a9}{}}%
Now that we have the results, we can visualize the two populations which
we have prepared on a simple scatter plot, showing results from the
ground state program in blue and results from the excited state
preparation program in red. Note: If the populations irregularly shaped
(not approximately circular), try re-running the notebook.

\hypertarget{efb2485c-6374-4015-8230-49a5e9ca27e4}{}
\hypertarget{0-and-1-discrimination}{%
\subsection{\textbar0\textgreater{} and \textbar1\textgreater{}
Discrimination}\label{0-and-1-discrimination}}

\hypertarget{5abe36be-5884-4104-85d2-efc9625b4d82}{}
\begin{Shaded}
\begin{Highlighting}[]
\NormalTok{gnd\_results }\OperatorTok{=}\NormalTok{ gnd\_exc\_results.get\_memory(}\DecValTok{0}\NormalTok{)[:, qubit]}\OperatorTok{*}\NormalTok{scale\_factor}
\NormalTok{exc\_1\_results }\OperatorTok{=}\NormalTok{ gnd\_exc\_results.get\_memory(}\DecValTok{1}\NormalTok{)[:, qubit]}\OperatorTok{*}\NormalTok{scale\_factor}
\NormalTok{plt.figure()}

\CommentTok{\# Plot all the results}
\CommentTok{\# All results from the gnd\_schedule are plotted in blue}
\NormalTok{plt.scatter(np.real(gnd\_results), np.imag(gnd\_results), }
\NormalTok{                s}\OperatorTok{=}\DecValTok{5}\NormalTok{, cmap}\OperatorTok{=}\StringTok{\textquotesingle{}viridis\textquotesingle{}}\NormalTok{, c}\OperatorTok{=}\StringTok{\textquotesingle{}blue\textquotesingle{}}\NormalTok{, alpha}\OperatorTok{=}\FloatTok{0.5}\NormalTok{, label}\OperatorTok{=}\StringTok{\textquotesingle{}state\_0\textquotesingle{}}\NormalTok{)}
\CommentTok{\# All results from the exc\_schedule are plotted in red}
\NormalTok{plt.scatter(np.real(exc\_1\_results), np.imag(exc\_1\_results), }
\NormalTok{                s}\OperatorTok{=}\DecValTok{5}\NormalTok{, cmap}\OperatorTok{=}\StringTok{\textquotesingle{}viridis\textquotesingle{}}\NormalTok{, c}\OperatorTok{=}\StringTok{\textquotesingle{}red\textquotesingle{}}\NormalTok{, alpha}\OperatorTok{=}\FloatTok{0.5}\NormalTok{, label}\OperatorTok{=}\StringTok{\textquotesingle{}state\_1\textquotesingle{}}\NormalTok{)}

\NormalTok{plt.axis(}\StringTok{\textquotesingle{}square\textquotesingle{}}\NormalTok{)}

\CommentTok{\# Plot a large dot for the average result of the 0 and 1 states.}
\NormalTok{mean\_gnd }\OperatorTok{=}\NormalTok{ np.mean(gnd\_results) }\CommentTok{\# takes mean of both real and imaginary parts}
\NormalTok{mean\_1\_exc }\OperatorTok{=}\NormalTok{ np.mean(exc\_1\_results)}
\NormalTok{plt.scatter(np.real(mean\_gnd), np.imag(mean\_gnd), }
\NormalTok{            s}\OperatorTok{=}\DecValTok{200}\NormalTok{, cmap}\OperatorTok{=}\StringTok{\textquotesingle{}viridis\textquotesingle{}}\NormalTok{, c}\OperatorTok{=}\StringTok{\textquotesingle{}black\textquotesingle{}}\NormalTok{,alpha}\OperatorTok{=}\FloatTok{1.0}\NormalTok{, label}\OperatorTok{=}\StringTok{\textquotesingle{}state\_0\_mean\textquotesingle{}}\NormalTok{)}
\NormalTok{plt.scatter(np.real(mean\_1\_exc), np.imag(mean\_1\_exc), }
\NormalTok{            s}\OperatorTok{=}\DecValTok{200}\NormalTok{, cmap}\OperatorTok{=}\StringTok{\textquotesingle{}viridis\textquotesingle{}}\NormalTok{, c}\OperatorTok{=}\StringTok{\textquotesingle{}black\textquotesingle{}}\NormalTok{,alpha}\OperatorTok{=}\FloatTok{1.0}\NormalTok{, label}\OperatorTok{=}\StringTok{\textquotesingle{}state\_1\_mean\textquotesingle{}}\NormalTok{)}

\NormalTok{plt.ylabel(}\StringTok{\textquotesingle{}I [a.u.]\textquotesingle{}}\NormalTok{, fontsize}\OperatorTok{=}\DecValTok{15}\NormalTok{)}
\NormalTok{plt.xlabel(}\StringTok{\textquotesingle{}Q [a.u.]\textquotesingle{}}\NormalTok{, fontsize}\OperatorTok{=}\DecValTok{15}\NormalTok{)}
\NormalTok{plt.title(}\StringTok{"0{-}1 discrimination"}\NormalTok{, fontsize}\OperatorTok{=}\DecValTok{15}\NormalTok{)}

\NormalTok{plt.show()}
\end{Highlighting}
\end{Shaded}

\includegraphics{562210eb779f62ada709af584769b0b70e0cd73c.png}

\hypertarget{23270964-b117-457f-8926-023ab76782d1}{}
\begin{Shaded}
\begin{Highlighting}[]
\NormalTok{exc\_2\_results }\OperatorTok{=}\NormalTok{ gnd\_exc\_results.get\_memory(}\DecValTok{2}\NormalTok{)[:, qubit]}\OperatorTok{*}\NormalTok{scale\_factor}
\NormalTok{plt.figure()}

\CommentTok{\# Plot all the results}
\CommentTok{\# All results from the gnd\_schedule are plotted in blue}
\NormalTok{plt.scatter(np.real(gnd\_results), np.imag(gnd\_results), }
\NormalTok{                s}\OperatorTok{=}\DecValTok{5}\NormalTok{, cmap}\OperatorTok{=}\StringTok{\textquotesingle{}viridis\textquotesingle{}}\NormalTok{, c}\OperatorTok{=}\StringTok{\textquotesingle{}blue\textquotesingle{}}\NormalTok{, alpha}\OperatorTok{=}\FloatTok{0.5}\NormalTok{, label}\OperatorTok{=}\StringTok{\textquotesingle{}state\_0\textquotesingle{}}\NormalTok{)}
\CommentTok{\# All results from the exc\_schedule are plotted in red}
\NormalTok{plt.scatter(np.real(exc\_1\_results), np.imag(exc\_1\_results), }
\NormalTok{                s}\OperatorTok{=}\DecValTok{5}\NormalTok{, cmap}\OperatorTok{=}\StringTok{\textquotesingle{}viridis\textquotesingle{}}\NormalTok{, c}\OperatorTok{=}\StringTok{\textquotesingle{}red\textquotesingle{}}\NormalTok{, alpha}\OperatorTok{=}\FloatTok{0.5}\NormalTok{, label}\OperatorTok{=}\StringTok{\textquotesingle{}state\_1\textquotesingle{}}\NormalTok{)}
\NormalTok{plt.scatter(np.real(exc\_2\_results), np.imag(exc\_2\_results), }
\NormalTok{                s}\OperatorTok{=}\DecValTok{5}\NormalTok{, cmap}\OperatorTok{=}\StringTok{\textquotesingle{}viridis\textquotesingle{}}\NormalTok{, c}\OperatorTok{=}\StringTok{\textquotesingle{}green\textquotesingle{}}\NormalTok{, alpha}\OperatorTok{=}\FloatTok{0.5}\NormalTok{, label}\OperatorTok{=}\StringTok{\textquotesingle{}state\_2\textquotesingle{}}\NormalTok{)}
\NormalTok{plt.axis(}\StringTok{\textquotesingle{}square\textquotesingle{}}\NormalTok{)}

\CommentTok{\# Plot a large dot for the average result of the 0 and 1 states.}
\NormalTok{mean\_gnd }\OperatorTok{=}\NormalTok{ np.mean(gnd\_results) }\CommentTok{\# takes mean of both real and imaginary parts}
\NormalTok{mean\_1\_exc }\OperatorTok{=}\NormalTok{ np.mean(exc\_1\_results)}
\NormalTok{mean\_2\_exc }\OperatorTok{=}\NormalTok{ np.mean(exc\_2\_results)}
\NormalTok{plt.scatter(np.real(mean\_gnd), np.imag(mean\_gnd), }
\NormalTok{            s}\OperatorTok{=}\DecValTok{200}\NormalTok{, cmap}\OperatorTok{=}\StringTok{\textquotesingle{}viridis\textquotesingle{}}\NormalTok{, c}\OperatorTok{=}\StringTok{\textquotesingle{}black\textquotesingle{}}\NormalTok{,alpha}\OperatorTok{=}\FloatTok{1.0}\NormalTok{, label}\OperatorTok{=}\StringTok{\textquotesingle{}state\_0\_mean\textquotesingle{}}\NormalTok{)}
\NormalTok{plt.scatter(np.real(mean\_1\_exc), np.imag(mean\_1\_exc), }
\NormalTok{            s}\OperatorTok{=}\DecValTok{200}\NormalTok{, cmap}\OperatorTok{=}\StringTok{\textquotesingle{}viridis\textquotesingle{}}\NormalTok{, c}\OperatorTok{=}\StringTok{\textquotesingle{}black\textquotesingle{}}\NormalTok{,alpha}\OperatorTok{=}\FloatTok{1.0}\NormalTok{, label}\OperatorTok{=}\StringTok{\textquotesingle{}state\_1\_mean\textquotesingle{}}\NormalTok{)}
\NormalTok{plt.scatter(np.real(mean\_2\_exc), np.imag(mean\_2\_exc), }
\NormalTok{            s}\OperatorTok{=}\DecValTok{200}\NormalTok{, cmap}\OperatorTok{=}\StringTok{\textquotesingle{}viridis\textquotesingle{}}\NormalTok{, c}\OperatorTok{=}\StringTok{\textquotesingle{}black\textquotesingle{}}\NormalTok{,alpha}\OperatorTok{=}\FloatTok{1.0}\NormalTok{, label}\OperatorTok{=}\StringTok{\textquotesingle{}state\_1\_mean\textquotesingle{}}\NormalTok{)}

\NormalTok{plt.ylabel(}\StringTok{\textquotesingle{}I [a.u.]\textquotesingle{}}\NormalTok{, fontsize}\OperatorTok{=}\DecValTok{15}\NormalTok{)}
\NormalTok{plt.xlabel(}\StringTok{\textquotesingle{}Q [a.u.]\textquotesingle{}}\NormalTok{, fontsize}\OperatorTok{=}\DecValTok{15}\NormalTok{)}
\NormalTok{plt.title(}\StringTok{"0{-}1{-}2 Discrimination"}\NormalTok{, fontsize}\OperatorTok{=}\DecValTok{15}\NormalTok{)}

\NormalTok{plt.show()}
\end{Highlighting}
\end{Shaded}

\includegraphics{fe109f094eeff6eb26f781b56c5651bfb58f03a6.png}

\hypertarget{c65fdc65-ada0-4418-921c-2c66cd502816}{}
\begin{Shaded}
\begin{Highlighting}[]

\end{Highlighting}
\end{Shaded}
